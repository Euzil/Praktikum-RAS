\documentclass{article}
\usepackage[utf8]{inputenc}
\usepackage{bbding}
\usepackage{amssymb}
\begin{document}
\Large
Robotik-Praktikum Übung1
\large
Youran Wang 719511 RAS


\section{Lineare Transformation}
a).
\[
\left[ \begin{array}{ccc}
    1.3  &  0       & 0  \\
    0 &  0.83 &  0 \\
    0 &  0 &  1  \\
\end{array}
\right]
\]
b).
\[
R_{y}= \left[ \begin{array}{ccc}
    cos(45) &  0       & sin(45)  \\
    0 &  1 &  0 \\
    -sin(45) &  0 &  cos(45)  \\
\end{array}\right] = \left[ \begin{array}{ccc}
    0.7071 &  0       & 0.7071  \\
    0 &  1 &  0 \\
    -0.7071 &  0 &  0.7071 \\
\end{array}
\right]
\]
c).
\[
 T_[1]=\left[ \begin{array}{ccc}
    0.9848  &  0       & 0.1736 \\
    0.1228  &  0.7071  & -0.6964 \\
    -0.1228 &  0.7071  & 0.6964  \\
\end{array}\right]\\
\]\[
T_[2]=\left[ \begin{array}{ccc}
    0.9848  &  0.1228  & 0.1228 \\
    0       &  0.7071  & -0.7071 \\
    -0.1736 &  0.6964  & 0.6964  \\
\end{array}\right]\\
\]
Es gibt Unterschied. Weil die 2 Rotationen sind Euler-Winkel und YPR-Winkel. Also intrinsisch oder extrinsisch ist. Es bestimmt, in welchem Koordinatensystem die Rotation ist.
\\
\\
Intrinsische Rotation ist im World-Koordinatensystem, das bedeutet die 2. Rotation wird noch um der Achse des World-Koordinatensystem machen.
\\
\\
Extrinsische Rotation ist im Roboter-Koordinatensystem, das bedeutet die 2. Rotation wird auch um der neue Achse machen.

\section{Homogene Koordinaten und Transformation}
a).
\[
\left[ \begin{array}{cccc}
     1.3 &  0      & 0  & 10\\
     0   &  0.83   & 0  & 0 \\
     0   &  0      & 1  & 0 \\
     0   &  0      & 0  & 1 \\
\end{array}
\right]
\]
b).
\[
T= \left[ \begin{array}{cccc}
         &         &    &   \\
         &  M      &    & t \\
         &         &    &   \\
     0   &  0      & 0  & 1 \\
\end{array}
\right]
\]
\[
M^{'}p^{'}=(M*p , 0)^{T}+(t,1)^{T}
\]
\[
\Rightarrow p^{'}=p*M+(t,1)^{'}
\]
\[
\Rightarrow p^{}=(p^{'}-(t,1)^{'})/M^{-1}
\]
\[
\Rightarrow
T^{-1}=\left[ \begin{array}{cccc}
         &         &    &   \\
         &  M^{-1}      &    & -M^{-1}t \\
         &         &    &   \\
     0   &  0      & 0  & 1 \\
\end{array}
\right]
\]
Punkt P macht Translation und Rotation mit Transformation T, und dann p' bekommt.
\\
Gegenseitig können wir auch den Punkt p'mit Transformation T' zurück machen, um den Punkt p anzunehmen.
\\
Deshalb beim Rotationsteil benutzen wir M',und Translationsteil ist -M't.

\newpage
\section{Eigenschaften von Vektoren}

a).
\[
||v|| = \sqrt{(v,v)} = \sqrt{1^{2} + (\sqrt{2})^{2} + 0^{2}} = \sqrt{3}
\]
b).
\[
\alpha=acos((v,w)/||v||*||w||)
\]
\[
(v,w)=1/\sqrt{3}
\]
\[
||v||= \sqrt{(v,v)}=1
\]
\[
||w||= = \sqrt{(w,w)}=1
\]
\[
\alpha=acos((1/\sqrt{3})/1)=acos(1/\sqrt{3})=55.25^{o} oder -124.75^{o}
\]

\section{Transformation von koordinatensystem}
a).
\[
T=\left[ \begin{array}{cccc}
     0.7071   &  -0.7071     &  0  &  4 \\
     0.7071   &  0.7071      &  0  &  3 \\
     0        &  0           &  1  &  0 \\
     0        &  0           &  0  &  1 \\
\end{array}
\right]
\]
b).
\[
O=(0,-5)
\]
\[
v=(3\sqrt{2},0)
\]
c).
Ja, wir können die benutztn. Die alte Koordinaten wird mit Transformation multipliziert.


\section{Eigenschatfen von Matrizen}
a).
\[
det=1.3*0.83*1+0+0-0-0-0=1.079
\]
b).
\[
R = \left[ \begin{array}{ccc}
    0.7071 &  0       & 0.7071  \\
    0 &  1 &  0 \\
    -0.7071 &  0 &  0.7071 \\
\end{array}
\right]
\]
\[
det(A-xE_{n}) = 0 
\]
\[
\Rightarrow
det\left[ \begin{array}{ccc}
        0.7071-\lambda & 0         & 0.7071  \\
        0              & 1-\lambda & 0 \\
        -0.7071        & 0         & 0.7071-\lambda  \\
\end{array}
\right]=0
\]
\[
(0.7071-\lambda)(1-\lambda)(0.7071-\lambda)-0.7071*(-0.7071)*(1-\lambda)=0
\]
\[
\lambda=1
\]

Eigenwerte ist \lambda=1

\[
\left[ \begin{array}{ccc}
        -0.2929        & 0   & 0.7071  \\
        0              & 0   & 0 \\
        -0.7071        & 0   & -0.2929 \\
\end{array}
\right]
\]
\[
x=(a,b,c)^{T}
\]
\[
(A-\lambda*E)*x=0
\]
\[
\left[ \begin{array}{ccc}
        -0.2929        & 0   & 0.7071  \\
        0              & 0   & 0 \\
        -0.7071        & 0   & -0.2929 \\
\end{array}
\right]*\left[ \begin{array}{c}
        a & b & c  \\
\end{array}
\right]=0
\]
\[
\Rightarrow  \begin{array}{c}
        -0.2929a+0.7071c=0 & -0.7071a-0.2929c=0 \\
\end{array}
\]
\[
\Rightarrow  \begin{array}{c}
        a=0 & b=beliebig & c=0 \\
\end{array}
\]
Eigenvektoren ist (0,b,0)^{T}

\end{document}
